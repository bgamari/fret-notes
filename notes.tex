\documentclass{article}
\usepackage{amsmath}
\title{Some notes on single-molecule FRET}
\author{Ben Gamari}

\newcommand{\emm}[1]{\ensuremath{_{#1_\mathrm{em}}}}
\newcommand{\exc}[1]{\ensuremath{^{#1_\mathrm{exc}}}}
% \i{em}{exc}
\newcommand{\I}[2]{\ensuremath{I\emm{#1}\exc{#2}}}
    
\begin{document}
\maketitle

F\"orster Resonance Energy Transfer (FRET) is a quantum mechanical
process due to the electromagnetic coupling between dipoles. Under
this interaction, an excited fluorophore (the donor) can transfer energy
to a nearby molecule (the acceptor) with given an appreciable mutual
spectral overlap. FRET is often used in the study of single-molecule
systems, where it enables the determination of reaction kinetics,
binding constants, and some classes of conformational dynamics.
By exciting the donor while observing photon emissions from both the
donor and accessor, the experimentalist can assess the efficiency of
the FRET coupling. Informed by further physical models, FRET
experiments can be interpretted to make strong, physically-relevant
inferences.

We define the FRET efficiency $E$ to be the quantum yield of this
energy transfer process. As such, we can write it as a function of the
system's transition rates,
\begin{equation}
  E = \frac{k_{ET}}{k_{ET} + \sum_i k_{R_i} + \sum_i k_{NR_i}}
  \label{Eq:fretRates}
\end{equation}
where,

\begin{tabular}{lcl}
  $k_{ET}$  & = & rate of donor to acceptor energy transfer \\
  $k_{D-R}$   & = & rate of donor fluorescence radiative decay \\
  $k_{D-NR}$  & = & rate of donor non-radiative decay \\
\end{tabular}

We define $k_{DA}$ to be the effective rate of donor relaxation in the presence
of the acceptor,
\[ k_{DA}^{-1} = k_{ET}^{-1} + \sum_i k_{D-R_i}^{-1} + \sum_i k_{D-NR_i}^{-1} \]
Eliminating $k_{ET}$ from \eqref{Eq:fretRates} gives the alternative expression,
\begin{equation}
  E = 1 - \frac{k_{DA}}{k_{D-R}} \label{Eq:fretEffRates}
\end{equation}
By relating the fluorescence lifetimes to the rates, $\tau = k^{-1}$,
we find a similar result in terms of this relevant physical parameter,
\begin{equation}
  E = 1 - \frac{\tau_{D}}{\tau_{DA}} \label{Eq:fretEffTau}
\end{equation}

Equivalently, we can give the FRET efficiency in terms of equilibrium
parameters of the system,
\begin{equation}
  E = \frac{1}{1 + \left( \frac{r}{R_0} \right)^6}
  \label{Eq:fretEff}
\end{equation}
where $R_0$ is the F\"orster radius,
\begin{equation}
  R_0^6 = \frac{9 c^4 Q_0 \kappa^2 J}{128 \pi^5 N_A n^4}
  \label{Eq:R0}
\end{equation}
Where $N_A$ is Avogadro's number, $Q_0$ is the quantum yield of the
donor in the absence of the acceptor, and $n$ is the index of
refraction of the solvent.

$J$ denotes the spectral overlap integral,
\[ J = \int f_D(\lambda) \epsilon_A(\lambda) \lambda^4 d\lambda \]
where $f_D$ is the normalized donor emission spectrum and $\epsilon_A$
is the acceptor's molar extinction coefficient.

An important but oft-overlooked ``constant'' of \eqref{Eq:R0} is
the so-called dipole orientation factor, $\kappa^2$,
\begin{equation}
  \kappa = \hat\mu_D \cdot \hat\mu_A - 3(\hat\mu_D \cdot \hat R_{DA}) (\hat\mu_A \cdot \hat R_{DA})
  \label{Eq:kappa}
\end{equation}

While measurement of the FRET efficiency is possible by way of the
fluorescence lifetime and \eqref{Eq:fretEffTau}, a more common
approach is by way of an intensity measurement. Here, one 
measures $k_{R}$ and $k_{ET}$ by observing fluorescence intensities
from the donor and acceptor fluorophores, $k^*_D$ and $k^*_A$. This
measurement, however, is fraught with artifacts.

Let us examine the processes responsible for the observed rates
$k^*_D$ and $k^*_A$ for the acceptor and donor respectively. Assume
that the fluorescence of interest originates from donor decay channel
$D-R_j$ and acceptor channel $A-R_k$. We first note that not every
excitation event will result in an emission photon due to the
non-radiative decay pathways $NR_i$. We call the efficiency of photon
product the quantum yield $\phi$,
\begin{align*}
  \phi_D & = \frac{k_{D-R_j}}{\sum k_{D-R_i} + \sum k_{D-NR_i}} \\
  \phi_A & = \frac{k_{A-R_k}}{\sum k_{A-R_i} + \sum k_{A-NR_i}} \\
\end{align*}

Furthermore, any real measurement will have less than perfect, and
often wavelength dependent, detection efficiency, $\eta < 1$.  Under
these considerations, we assuming a donor excitation rate $k_{exc}$ to
find fluorescence rates $k^*_D$ and $k^*_A$,
\begin{align*}
  k^*_D & = k_{exc} ~ \eta_D ~ \frac{k_{D-R_j}}{k_{ET} + \sum k_{D-R_i} + \sum k_{D-NR_i}} \\
        & = k_{exc} ~ \eta_D ~ \phi_D ~
           \frac{\sum k_{D-R_i} + \sum k_{D-NR_i}}{k_{ET} + \sum k_{D-R_i} + \sum k_{D-NR_i}} \\
        & = k_{exc} ~ \eta_D ~ \phi_D ~ (1 - E) \\
  k^*_A & = k_{exc} ~ \eta_A ~ \phi_A ~ E \\
\end{align*}

From these rates, we can construct an observed FRET efficiency (also
known as the proximity ratio), $E^*$,
\begin{align*}
  E^* & = \frac{k^*_A}{k^*_A + k^*_D} \\
      & = \left[ 1 + \left(\frac{\phi_A~\eta_A}{\phi_D~\eta_D}\right)^{-1} (E^{-1} - 1) \right]^{-1} \\ \label{Eq:proxRatio}
\end{align*}
We denote the quantity $(\phi_A~\eta_A / \phi_D~\eta_D)$ as $\gamma$,
which represents the relative sensitivity of our experiment to the two
fluorophores. While the constituents of $\gamma$ are usually
prohibitively difficult to determine, correction for $\gamma$ as a
whole is quite easy given proper experimental design. From
\eqref{Eq:proxRatio}, we can derive the corrected FRET efficiency $E$,
\begin{equation}
  E = \frac{k^*_A}{k^*_A + \gamma k^*_D}
  \label{Eq:gammaFretEff}
\end{equation}
In addition to $\gamma$ correction, there are two further corrections that
must be routinely made during FRET data analysis. The first of these
is spectral crosstalk between the acceptor and donor channels. This
artifact is largely due to the imperfect spectral filtering of
dichroic mirrors, resulting in leakage of donor photons into the
acceptor channel. Moreover, all experiments must deal with a constant
background on top of the measured signal. Accounting for these two effects
yields an estimate for the true observed intensity,
\begin{align*}
  I_D & = k^*_D - \alpha~k^*_D + I_{D-BG} \\
  I_A & = k^*_A + \alpha~k^*_D + I_{A-BG} \\
\end{align*} 
where $I_{BG}$ are the background count rates and $\alpha$ is the crosstalk
parameter. Propagating these through our estimator for the FRET
efficiency, we find the background, cross-talk, and $\gamma$ corrected
FRET efficiency to be,
\[
  \hat E = \frac{E^* ~ \left(1 + \gamma ~ (\beta_A + \beta_D)\right) - \gamma~\beta_A - \alpha}
    {E^* ~ (1 - \gamma) + \gamma - \alpha}
\]
where,
\begin{align*}
  \beta_A & = \frac{I_A}{k_{exc}~\eta_A~\phi_A} \\
  \beta_D & = \frac{I_D}{k_{exc}~\eta_A~\phi_A} \\
\end{align*}
Clearly, the dependence on $k_{exc}~\eta_A~\phi_A$ poses a difficulty
as these are all unobserved quantities. One way to approach this would
be to ?.

\section{Determination of $\gamma$}
While we can now properly correct for $\gamma$ by way of
\eqref{Eq:gammaFretEff}, this tool is useless unless we can determine
the value of $\gamma$. The most obvious approach here is to examine a
population with a known FRET efficiency to derive $\gamma$, which one
can then use for correction. For this reason, the meticulous
experimentalist will always take data on sample lacking acceptor
fluorophores. As $\gamma$ depends upon instrumental state
(e.g. alignment, detector response), it is important to run these
calibration experiments often. Moreover, since the quantum yield of a
fluorophore depends strongly on its environment, it is wise to use the
specimen under study for these calibrations, with the acceptor
fluorophore either desensitized or not present, of course.

Lacking a dedicated donor-only experiment, one can often also
exploit the donor-only population usually present after sample
preparation as the calibration sample.

\section{Alternating Laser Excitation}
While FRET alone *, the information content in an experiment can be
greatly enhanced through use of Alternating Laser Excitation (ALEX) as
proposed by Kapanidis, {\it et al.} One of the great difficulties in
the interpretation of FRET results is the difficulty of resolving low
FRET efficiencies. Due to considerations of sample preparation,
photobleaching, and *, a substantial fraction of the molecules
measured in a typical FRET experiment will have no acceptor
fluorophore.

While distinguishing this population from a low-FRET state is nearly
impossible under traditional FRET, ALEX uses direct excitation of the
acceptor to probe its existence. This alternation is carried out at a
timescale far shorter than the specimens' diffusion time, giving
access to an additional axis of the typical FRET efficiency histogram.

In the following discussion, we adopt a notation similar to that of
Kapanidis, {\it et al}. Let \I{A}{D} denote the intensity of emission
into the acceptor channel under donor excitation. Under this notation,
we define our existing notation $I_A$ and $I_D$ as,
\begin{align*}
  I_A & = \I{A}{A} + \I{A}{D} \\
  I_D & = \I{D}{A} + \I{D}{D} \\
\end{align*}
Therefore, our familiar FRET efficiency is given by,
\[ E = \frac{\I{A}{D}}{\I{A}{D} + \gamma \I{D}{D}} \]
Moreover, we can now define a quantity which we call the
stoichiometry, $S$, which characterizes the ratio of acceptor
fluorophores to donor fluorophores,
\[ S = \frac{I\exc{D}}{I\exc{D} + I\exc{A}} \]

While ALEX allows us to separate previously unseparable populations,
it is no less susceptible to cross-talk, background, and $\gamma$
errors. Correcting $S$ in the same manner as we did $E$ earlier, we
find that ?.

\end{document}